\chapter{Requirements Analysis}
In this chapter, we will discuss the process of requirements analysis. Requirements have two aspects: user requirements and system requirements. User requirements reflect what kind of service the software is expected to provide to its users. They can usually be an abstract description of the software. System requirements are more detailed descriptions of the functionalities, services and constraints of the software system. Requirements analysis is usually the very first step of software engineering. It is that important because it will be very difficult to change the requirements in the next stages of software development. Failure in requirements analysis can lead to the consequence that requirements cannot be satisfied or the software cannot be delivered in time.

There are various approaches to requirements analysis \cite{kotonya1998requirements}. Since the requirements for our software are comparably simple and users are well accessible, we decided to adopt the interview approach. The interviewee was the supervisor of the thesis, who would also be a user and admin of the software. The interviews were either in form of emails or face-to-face talks. We had several rounds of interviews. After each interview, we updated the requirements and raised new questions. We did this in an iterative way until we confirmed the requirements were complete. We might want to add more functionalities in the later development stages. But no major modifications would be done on the requirements. We formulated the requirements with natural language.

\section{User Requirements}

To make an easy-to-use software, we have to design a very friendly graphical user interface. Different UI components have to be well organized. They should be simple and intuitive, with underlying logic hidden from the users. One of our goals is to minimize users' operations on the GUI.

The goal of the project is to develop a graphical user interface based software that aids in implementation of the digital configuration interface for AISC chips. It shall allow chip designers to define the system blocks, different types of registers and signals, initial values of signals, and signal-register mappings, and to write different document items of different kinds, including pure text, mathematics equations, tables and images, on different hierarchy levels using a friendly dialog. The software shall be able to display defined system blocks, registers, signals, signal-register mappings, chip designers, register pages and document items etc. in a navigable way. After definition of the chip is complete, the software shall be able to generate working VHDL and LaTeX source code. The user interface shall be carefully designed and hide as much details as possible from the chip designers. To implement the GUI, an appropriate framework should be selected.

The software shall allow chip designers to work simultaneously on different system blocks of the same chip. All inputs given from different users shall be stored in the same well-designed SQL database running on an appropriate database management system.

For data security, user access control and a login mechanism shall be implemented. A basic role model shall be designed, such that standard users are distinguished from the admins in both database or software level and project level permissions. To make the software more reliable, sanity checks must be implemented especially before database operations. A logger shall be implemented, and user activities and database operations shall be logged.

\section{System Requirements}
\subsection{Main Window}
\begin{itemize}
\item The central component of the GUI is a main window.
\item The main window shall contain two components: the navigator and the working area.
\item The navigator shall be a tree view in a CHIP-BLOCK-REGISTER-SIGNAL structure. It shall allow users to search for components of a given pattern and highlight the matching components. The working area shall contain a chip editor on the left and a document editor on the right. The two editors can be switched on or off.
\item The chip editor view shall be a stacked widget containing two pages: the chip level page and the block level page. The chip level page shall display read-only information about the basics of the chip, system blocks, chip designers and register pages. The block level page of the chip editor shall contain two tabs: one for registers and the other for signals. On the signal tab there should be two tables showing signals of the current system block and signal-register mappings of the current signal. The register tab is organized in a similar way.
\item The document editor view shall contain a table showing the list of document items, and a document editing area. The editing area is hidden. When users add a document item or edit an existing one, the editing area shall be displayed.
\item The working area shall respond to changes in the current item in the navigator. When the current item changes, the chip editor shall switch to either the chip level or the block level page. If in the block level page, either the register or signal tab is set as current. Then, the corresponding information shall be displayed on the chip editor. At the same time, the documents regarding this item shall be displayed, if the document editor page is switched on.
\item Unless otherwise noted, below each table in the chip editor and the document editor, there shall be an \textbf{Add} and a \textbf{Remove} button. The users shall be allowed to add a new item, remove or edit an existing item. 
\item At the top of the main window should be a menu bar containing actions on users, chips, views, exports and so on.
\end{itemize}
\subsection{Chips and Documents}
\begin{itemize}
\item The software shall allow users to edit chips and documents in dialogs. Each dialog is designed for a specific use case or functionality, such as adding a new system block. The dialogs must provide appropriate components for users to put in information.
\item The software shall allow users to add new chips or edit exiting chips. It should allow users to put in the chip name, register width, address width, and select whether the chip is MSB first.
\item The software shall allow users to add new system blocks or edit existing ones. It should allow users to put in the block name, block abbreviation, start address in hexadecimal format and select the responsible designer. The start address should fit the chip's address width.
\item The software shall allow project admins to add new chip designers or edit existing ones. It should allow admins to select a designer out of all users in the database, and select the designer's project role.
\item The software shall allow users to add new registers to the current system block or edit existing ones. It should allow users to put in the register name and select a register type out of all register types in the database.
\item The software shall allow users to add new signals to the current system block or edit existing ones. It should allow users to put in the signal name, width, select the signal type, and check whether to add a port for this signal. If the signal is a register signal, i.e. it shall be mapped to registers, the dialog shall allow users to select a register type. When the register type is writable, it shall also allow users to define the initial value in hexadecimal format.
\item The software shall allow users to map signal bits and register bits in an efficient and convenient way. A signal bit can only be mapped to one register bit, and the other way around. 
\item The software shall allow users to add documents to the currently selected item in the navigator or edit existing ones. It shall allow users to select a document type. For a text document, it shall allow users to put in text data. For an image document, it shall allow users to select an image and edit its caption and width. For a table document, it shall allow users to edit the table contents in a table widget and put in its caption. The number of rows and columns of the table shall be adjustable.
\item The software shall allow users to add register pages or edit existing ones. It shall allow users to select an available control signal and specify the page name and a valid page count. It shall allow users to select available registers in a convenient way. The control signal must not be mapped to any registers in the page it controls. A signal can control at most one register page block.
\end{itemize}
\subsection{User Access and Authentication}
\begin{itemize}
\item The software shall allow database admins to create users conveniently.
\item A login mechanism shall be implemented. The main window of the software shall open only when the users provide correct username and password through the login dialog.
\item The software shall allow users to select an existing chip to open. The chips shall be listed in a table containing basic information about them.
\item Each user shall be assigned a database role. The database roles shall be defined in the database, determining whether the users can or cannot add or remove projects or users. 
\item Each project shall have at least one chip designers. Chip designers shall be assigned a project role. The project roles are defined in the database, determining whether the users can add or remove system blocks, chip designers, register pages, registers, signals, signal-register mappings, and documents belonging to a specific chip item (i.e. a system block, a register etc). 
\item The UI shall be adjusted according to the the database role and the project role. For example, some widgets might be disabled if the user is not eligible for it.
\item When the design of a certain chip is complete, the chip owner can freeze the chip. In this case, everything about the chip, except documentation, is frozen, and cannot be edited anymore. In such case, the UI shall disable relevant UI components. It shall allow the chip owner to unfreeze the frozen chip.
\end{itemize}
\subsection{SPI Interface and Documentation generation}
\begin{itemize}
\item The software shall be able to generate a VHDL SPI interface. It shall allow users to select SPI interface templates and configure the output before generation.
\item The software shall be able to generate a LaTeX documentation. It shall allow users to configure the documentation, for example, to select which system blocks to include in the documentation.
\end{itemize}
\subsection{Reliability}
\begin{itemize}
\item Error handling for database operations shall be implemented.
\item All write operations on the database and certain user activities shall be logged for error diagnosis purpose.
\item The software must check if each chip editing or document editing operation, such as adding a new system block or removing a register, is legal. The software shall check e.g. names of the signals/registers/system blocks/users etc, the initial value of a signal, the start address of a system block and so on.
\item Before generating the VHDL source code, the software shall check whether the configuration file contains all necessary predefined variables, whether address blocks of different system blocks overlap or exceed the address width, and whether the signal-register mappings are valid.
\end{itemize}

\section{Conclusion}
Requirement engineering is the very first step of software development. It is of crucial importance, since it is very costly to change the requirements in the following stages. In case of a comparatively small software project, we might be able to define all requirements before the following steps. However, when the project becomes more complex, it will be difficult to write a complete requirement list. There are cases where new requirements have to be added and existing requirements have to be modified.

In our practice, we first made a draft containing the most fundamental requirements. We then made a prototype to help us determine requirements in particular regarding the UI. Other requirements regarding data security and reliability can be comparably easily determined without a prototype.