\pagestyle{empty}
% set textheight so the date also fits when Tittles get bigger
\enlargethispage{3cm}
% set linespacing to single line.
{

\setstretch{1}%
	
\begin{center}
	\textbf{Eidesstattliche Versicherung}
\end{center}
\begin{tabular}{cp{5.2cm}c}
\hspace{5cm}	& & \hspace{4cm} \\ 
\IASAuthor 		& & \IASMatNr \\\cline{1-1} \cline{3-3}
Name   & & Matrikelnummer \\
				& &(freiwillige Angabe) \\
\end{tabular}
\par
Ich versichere hiermit an Eides Statt, \hfill \newline 
dass ich die vorliegende \IASSubject \- mit dem Titel \hfill 
\par
\ThesisTitleGerman 
\par
\ThesisTitleEnglish 
\par
selbständig und ohne unzulässige fremde Hilfe erbracht habe. Ich habe keine anderen als die angegebenen Quellen und Hilfsmittel benutzt. Für den Fall, dass die Arbeit zusätzlich auf einem Datenträger eingereicht wird, erkläre ich, dass die schriftliche und die elektronische Form vollständig übereinstimmen. Die Arbeit hat in gleicher oder ähnlicher Form noch keiner 
Prüfungsbehörde vorgelegen.
\par
% bring the next text down to the bottom so variation of title lenght don´t change the position of this text
\begin{tabular}{cp{5.2cm}c}
	\hspace{5cm}				& & \hspace{4cm} 	\\ 
	Aachen, \IASSubmissionDate 	& &   				\\ \cline{1-1} \cline{3-3} 
	Ort, Datum     				& & Unterschrift 	\\
\end{tabular}
% bring the next text down to the bottom so variation of title lenght don´t change the position of this text
\par 
\vspace*{\fill} 
\par
\textbf{Belehrung:} 

\footnotesize 

\textbf{§~156~StGB: Falsche Versicherung an Eides Statt}

\setstretch{1.2}%
Wer vor einer zur Abnahme einer Versicherung an Eides Statt zuständigen Behörde eine solche Versicherung falsch abgibt oder unter Berufung auf eine solche Versicherung falsch aussagt, wird mit Freiheitsstrafe bis zu drei Jahren oder mit Geldstrafe bestraft.

\textbf{§~161~StGB: Fahrlässiger Falscheid; fahrlässige falsche Versicherung an Eides Statt }

(1) Wenn eine der in den §§~154 bis 156 bezeichneten Handlungen aus Fahrlässigkeit begangen worden ist, so tritt Freiheitsstrafe bis zu einem Jahr oder Geldstrafe ein. \newline
(2) Straflosigkeit tritt ein, wenn der Täter die falsche Angabe rechtzeitig berichtigt. Die Vorschriften des §~158 Abs. 2 und 3 gelten entsprechend.

\normalsize 
Die vorstehende Belehrung habe ich zur Kenntnis genommen:\\
\begin{tabular}{cp{5cm}c}
	\hspace{5cm}				& & \hspace{4cm} 	\\ 
	Aachen, \IASSubmissionDate 	& &   				\\ \cline{1-1} \cline{3-3} 
	Ort, Datum     				& & Unterschrift 	\\
\end{tabular}
}%END set linespacing to single line.
\cleardoublepage